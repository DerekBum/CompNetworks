%! Author = tulchin
%! Date = 26.02.2023

% Preamble
\documentclass[11pt]{article}
\usepackage[utf8x]{inputenc}
\usepackage[english,russian]{babel}
\usepackage{cmap}
\usepackage{graphicx}
\usepackage{gensymb}
\graphicspath{{./images/}}
\DeclareGraphicsExtensions{.pdf,.png,.jpg}

% Packages
\usepackage{amsmath}
\usepackage{amssymb}
\usepackage{amsfonts}
\usepackage{dsfont}
\usepackage[normalem]{ulem}
\usepackage{geometry}
\geometry{
    a4paper,
    top=25mm,
    right=15mm,
    bottom=25mm,
    left=15mm
}

% Document
\begin{document}

{\bf Задание 1}\\

Рассмотрим первые несколько пакетов при передаче.
В момент времени $\frac{L}{R}$ первый пакет достигнет первого маршрутизатора, и он начнет передавать его дальше.
Но в то же время от источника начнет передаваться второй пакет.

Таким образом, в момент времени $\frac{2L}{R}$ первый пакет достигнет второго маршрутизатора, второй -- первого, а третий начнет передаваться источником.

В момент времени $N \frac{L}{R}$ первый пакет достигнет приемника, второй пакет будет на последнем маршрутизаторе, третий -- на предпоследнем и так далее.
Всего еще надо дождаться $(P - 1)$ пакета.

Последний пакет достигнет первого маршрутизатора в момент времени $P \frac{L}{R}$ и после этого отправится до второго, третьего и тд.

Значит, потрачено будет $(N+P-1) \frac{L}{R}$ времени.\\

{\bf Задание 2}\\

Посчитаем скорости передачи данных (в байтах) по каналам: $v_1 = \frac{200 \cdot 1024}{8} = 25600$ (байт в секунду), $v_2 = 393216$, $v_3 = 262144$.
Размер пакета возьмем минимальный возможный -- 64 байта.

Заметим, что скорость второго канала намного быстрее скорости первого.
Это значит, что при получении текущего пакета вторым каналом, все предыдущие уже будут по нему отправлены (и дойдут до третьего канала).
То есть, текущий пакет без промедлений отправится по второму каналу.

Посчитаем за сколько пакет проходит первый канал, а также второй и третий вместе.
Первый: $\frac{64}{25600} = 2.5$ мксек.
Второй и третий: $\frac{64}{393216} + \frac{64}{262144} \approx 0.41$ мксек.

Из этого можно сделать вывод: как только $(i + 1)-$ый пакет дойдет до второго канала, $i-$ый уже будет доставлен на приемник.
Таким образом, последний пакет определяет все время доставки: ему надо дождаться прохождения по первому каналу, а затем пройти по второму и третьему.

Значит, если $P$ -- количество пакетов, $L$ -- их размер, то потратим мы суммарно времени $\frac{PL}{v_1} + \frac{L}{v_2} + \frac{L}{v_3} \approx 204.8$ сек.
Это примерно 3 минуты 25 секунд.\\

{\bf Задание 3}\\

Заметим, что пользователи независимы друг с другом.
Значит, можно построить модель: в каждый момент времени пользователь с вероятностью 0.2 активен (равен 1).
Если пользователь неактивен, то он равен 0.
И надо оценить вероятность того, что сумма пользователей $\geq 12$.

Также можно заметить, что больше 20 пользователей передавать данные не могут (2 Мбита / 100 Кбит).

Мы можем воспользоваться теоремой Пуассона: $P(S_n = k) \longrightarrow \frac{\lambda^k}{k!} e^{-\lambda}$, где $S_n$ -- сумма по исходам, $\lambda = np = \frac{60}{5} = 12$.
$P(S_n \geq 12)$ можно выразить как $P(S_n = 12) + P(S_n = 13) + ... + P(S_n = 20)$.

Таким образом, $P(S_n \geq 12) \approx e^{-12} \cdot \left( \frac{12^{12}}{12!} + \frac{12^{13}}{13!} + \frac{12^{14}}{14!} + ... + \frac{12^{20}}{20!} \right) \approx 0.526805$.\\

{\bf Задание 4}\\

Воспользуемся результатом первой задачи и посчитаем задержку передачи файла.
$N = 3$, $P = \frac{X}{S}$:

\[d(S) = \left(2 + P \right) \cdot \frac{L}{R} = \left(2 + \frac{X}{S} \right) \cdot \frac{80 + S}{R} \]

И нам надо найти минимум этой функции.
Найдем ноль производной:

\[ \frac{\partial d}{\partial S} = \frac{2 \left( S^2 - 40 X \right)}{R \cdot S^2} = 0 \Longrightarrow S^2 = 40X \Longrightarrow S = \sqrt {40X} \]

Ну и понятно, почему мы нашли минимум, а не максимум.
При маленьких $S$ первый множитель стремится к $\infty$, при больших -- второй.
При этом оба множителя ограничены снизу какими-то константами.\\

{\bf Задание 5}\\

    Задержка передачи равна $\frac{L}{R}$.
    Тогда сумма задержек:
    \[ \frac{L}{R} + \frac{I \cdot L}{R(1 - I)} = \frac{L}{R - a \cdot L} \]

    Заметим, что формула работает если $I < 1 \Longleftrightarrow aL < R $.
    Иначе задержка равна $\infty$.\\

    Немного распишем формулу:
    \[\frac{L}{R - a \cdot L} = \frac{L / R}{1 - a L / R} = \frac{x}{1 - ax} \]
    если взять $x = \frac{L}{R}$.

\end{document}